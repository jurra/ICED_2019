%% Sample-Paper.tex
%% V1.0
%% Developed for Authors to create their manuscript.
%%
%% This file uses the coding defined in the LaTex template ICED-Paper.cls

\documentclass{ICED-Paper}%%%%where ICED-Paper is the template name

%The authors can define any packages after the \documentclass{ICED-Paper} command.

%\usepackage{algorithmic} for describing algorithms
%\usepackage{subfig} for getting the subfigures e.g., "Figure 1a and 1b" etc.

%The author can find the documentation of the above style file and any additional
%supporting files if required from "http://www.ctan.org"

% *** Do not adjust length that controls the trim size and margins ***

\begin{document}

%The title of the paper, names of authors, affiliations of authors, abstract
%and keywords for the paper \textbf{will be produced automatically} from the
%ConfTool conference management system, based on the data that you enter in
%the ConfTool. So, don't include those details in this document.

What about figure \ref{fig1}?

\begin{figure}
\centering{\includegraphics{Fig1.eps}}% Replace "fig1" with appropriate figure file name
\caption{Figure caption here\label{fig1}}
\end{figure}

Text here.

\begin{table}
\processtable{Table title here\label{tab1}}
{\tabcolsep=2pc\begin{tabular}{|l|c|}%Number of columns has to be defined here
\hline
Table text here & Table text here\\%Table body
\hline
Table text here & Table text here\\%Table body
\hline
\end{tabular}}{Table footnote here}
\end{table}

\section{Introduction}
\iffalse
    here we have to define what is the current state of understanding this phenomenon in the context of design studies and research,
    Open Source is an example of peer production in the sense that there is distributed and not centrally controled, peer to peer basis
    p2p collaborative model, management of the collaboration process, 
\fi

This is a citation \cite{cite_key1}, because wahtever this other person \cite{cite_key2} it is fine.

\subsection{The benefits of studying open source projects for design theory and methodology}

\subsection{Research goals}

\subsubsection{Understanding design activity in the context of peer production}

\section{Methodology}

\subsection{Specifying free based peer production}

\subsection{An overview of  the Activity Theory framework}

\subsection{Experimental design}
\iffalse
    here I explain the case selected, the rationale behind this selection, as well as the materials used, but also specific questions to answer the general questions,
\fi

\section{Findings}
\subsection{Essential artifacts in free based peer production activities}

\subsection{The role of users within free based peer production}

\subsubsection{Projects led by lead users}

\subsubsection{Community composition}

\subsection{The continuous expansion of activity objects in free based peer production communities}

\subsubsection{Integrated product extension}

\subsubsection{Product derivatives and variations}

\section{discussion}
\subsection{Domain specificities of software and hardware}

\subsection{Implications for design research and methodologies}

\subsection{Recommendations for further studies}

\section*{References}

\bibliographystyle{unsrtnat}

  \bibitem{p2pEconomy}Bauwens, M. (2005). The political economy of peer production. CTheory, 12-1.

  \bibitem{ConcreteMath}
  R.L. Graham, D.E. Knuth, and O. Patashnik, \emph{Concrete
  mathematics}, Addison-Wesley, Reading, MA, 1989.



\end{thebibliography}
\section*{Acknowledgments}

Acknowledgments text here.

\appendix

\section*{Appendix}

Appendix text here.

\end{document}
