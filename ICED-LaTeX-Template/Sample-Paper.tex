%% Sample-Paper.tex
%% V1.0
%% Developed for Authors to create their manuscript.
%%
%% This file uses the coding defined in the LaTex template ICED-Paper.cls

\documentclass{ICED-Paper}%%%%where ICED-Paper is the template name

%The authors can define any packages after the \documentclass{ICED-Paper} command.

%\usepackage{algorithmic} for describing algorithms
%\usepackage{subfig} for getting the subfigures e.g., "Figure 1a and 1b" etc.

%The author can find the documentation of the above style file and any additional
%supporting files if required from "http://www.ctan.org"

% *** Do not adjust length that controls the trim size and margins ***

\begin{document}

%The title of the paper, names of authors, affiliations of authors, abstract
%and keywords for the paper \textbf{will be produced automatically} from the
%ConfTool conference management system, based on the data that you enter in
%the ConfTool. So, don't include those details in this document.
\iffalse
What about figure \ref{fig1}?

\begin{figure}
\centering{\includegraphics{Fig1.eps}}% Replace "fig1" with appropriate figure file name
\caption{Figure caption here\label{fig1}}
\end{figure}

Text here.

\begin{table}
\processtable{Table title here\label{tab1}}
{\tabcolsep=2pc\begin{tabular}{|l|c|}%Number of columns has to be defined here
\hline
Table text here & Table text here\\%Table body
\hline
Table text here & Table text here\\%Table body
\hline
\end{tabular}}{Table footnote here}
\end{table}
\fi

\section{Introduction}
\iffalse
    here we have to define what is the current state of understanding this phenomenon in the context of design studies and research,
    Open Source is an example of peer production in the sense that there is distributed and not centrally controled, peer to peer basis
    p2p collaborative model, management of the collaboration process,
\fi

This is a citation \cite{RepRap}, because wahtever this other person \cite{vonHippel_1} it is fine.

\subsection{The benefits of studying open source projects for design theory and methodology}

\subsection{Research goals}

\subsubsection{Understanding design activity in the context of peer production}

\section{Methodology}
This work attempts to capture significant aspects of peer to peer dynamics in the context of open source projects. To map these contexts we use Activity Theory framework[REFERENCE]. This framework allowed us to capture patterns of behavior and collaborative dynamics similarities and differences, beyond the specificities of what tools or systems are used, how tools reflect intagible aspects of socio-technical systems like values, motivations, community rules as well as roles within collaborative dynamics.
\subsection{Specifying free based peer production}
- [ ] Peer to peer happens everywhere, but open collaborative peer to peer means that also unknown peers, in many cases directed unrelated with other peers geographically, ....
\subsection{An overview of  the Activity Theory framework}
This research uses the expanded Activity Theory framework developed by Engestrom [REFERENCE]. This framework provides a consistent model to analyze collective processes like product design and development as part of a social context. This approach considers tools and procedures as part of a broader set of human interactions.
\subsection{Experimental design}
\iffalse
    here I explain the case selected, the rationale behind this selection, as well as the materials used, but also specific questions to answer the general questions,
\fi
We describe at a Meso level these projects as activity  systems that evolve overtime, and how their community composition, activity objects, as well as roles, rules, and artifacts reflect such evolution. Furthermore we attempt to describe how the outcomes of the activity system develop, along with the activity structure.

The two sets of cases (three for software, and three for hardware) are analyzed independently with respect to their own differences and similarities considering the different components of the Activity System model.
- What is the main object of the activity
- What is the main outcome(s) or product(s) of such activity
- What rules and tools they use to organize their collaboration
- What interesting aspects stand out in the case with regard to the outcomes, motivations, expectations as well as the collaboration process.



- Case study of diverse open source software and hardware projects with different characteristics:
We selected projects that present basic similarities with regard to basic freedoms mentioned in the previous section:
- For both sets of cases there is considerable difference in scope, product architecture, documentation, participation,tools but also in age. These set allows us to verify how time reflects in project maturity, adoption of modern practices, but also consistencies regardless of these differences.
- For the software projects there is a main similarity: they all use git based peer review tools.

### Materials
- Interviews and talks from founders of free software and open source movement.
- Forums of different projects like RepRap.
- Project and product documentation
- The Authors familiarity with certain open source projects.
- The availability of data like discussions, documentation, and resources to pursue this study.

### Procedure

Later on key differences and similarities between the two sets are described and discussed considering also the same questions points.

\section{Findings}

In this section we expose some prominent characteristics of the peer to peer interactions using the Activity Theory framework. We study at a very general level how peer to peer relations and interactions are reflected on the artifacts, object, subject, community, rules and division of labor components of the Activity System model.

\subsection{Essential artifacts in free based peer production activities}
From an Activity Theory stand point we can identify many artifacts within open source projects based on the different processes that take place. Mapping out all these artifacts is beyond the scope of this work. Nevertheless there are specific artifacts that are essential to enable an open and horizontal peer to peer collaborative process, or what we have been calling peer production artifacts:
- **Foundational artifacts**. These kind of artifacts describe in broader terms the main rules of the game for open source projects. Among these artifacts we can find perhaps the most important which is the **license** under which the content(source) is released (open) to the public.**Licenses** describe in legal terms what can and cannot be done with the source code ranging from ethical issues to commercial and distribution issues. They also often describe how to give credit to authors, and properly
release new improvements, but also distribute the content and more. It is interesting to notice that since the FREE Software licenses were created, every open source project has adopted or created similar licenses to distribute and share code, but also visual images as well as CAD files. There are other artifacts the specify how to contribute to the project often called *contributing guidelines*, but also the code of conduct, and instructions on how to participate.
- **Peer content production artifacts** There are variations in the way content creation takes place, but ultimately it requires of a web service, where other peers can access and modify the content created collaboratively. These ranges from dedicated git based services like github, or gitlab (Arduino, ), but also wikis.
- **Peer review artifacts** Like peer content production, peer review relates to a more operational level within the activity system. In the case of software, version control of the code is essential. This can be implemented also in different ways, but the standard and widely used system for version control at the moment in the software industry is git, for several reasons that go beyond our scope. Other tools for peer review such as from mailing lists, forums, and chats were found in both software and hardware cases.

\subsection{The role of users within free based peer production}
\subsubsection{Community based projects led by lead users}

Linus Torvalds, the creator of the Linux kernel, on an interview reveals:
> Every single project I haved worked on was for something I needed.

**We have found in our study that all the open source projects have been developed by lead users that have had some level of dissatisfaction with previous solutions** Lead users according to...

All these projects have started because at a certain point users with particular needs and problems, haven't found or have access to the solution they have been looking for to their problems. Moreover these *lead users* have had the capacities to build a new product, and enabled eventually that other users participate in developing, and improving the open source product in new directions.

For instance Linus Torvalds developed the Linux Kernel partly because he needed, partly because he enjoyed programming, he says that he never imagined Linux would evolve into what it is today. Richard Stallman the father of the FREE/Software movement and the GNU/Linux Project, refused to accept that computer scientists, programmers and users in general were not able to do study the source code.
Josef Prusa, a main contributor in the RepRap community explains in an interview:
>  originally got into 3D printing because I was into music, and I started to build my own MIDI controllers. I needed all sorts of little knobs and faders. So that’s how I found 3D printing. I started to build one myself, but it took so much time and so many parts that I eventually started to make it simpler. I started to improve it and give back, and so that’s how the Prusa Simplified Mendel came to the world.

**Open Source Ecology** was also motivated by the difficulties and circumstances that Marcin Jakubowski experienced after starting a farm:
> I bought a tractor then it broke, I paid to get it repaired, then it broke again....I realized that the truly appropriate, low-cost tools that I needed to start a sustainable farm....just didn't exist.

In the case of Git and Gitlab the original motivation was again to solve problems meeting specific requirements. Git started as a tool for version control on the Linux project. Version control is an essential artifact in the context of software.
>...we were in this bad spot where we had thousands of people
who wanted to participate, but in many ways, I was the kind of break point,where I could not scale to the point where I could work
with thousands of people. So Git is my second big project which was only created for me to maintain my first big project.(Linux)

Similarly Gitlab was started based on a need to collaborate on software development teams by a software developed. In the case of Arduino the main motivation was to create simple and low cost tools for creating digital projects by non-engineers in the an academic context. Before launching the project, the main users teachers and students, had difficulties because of not having a low cost solutions to teach electronics. Before Arduino, it was very expensive for students to work with microcontrollers during their studies.


\subsubsection{Community composition}

\subsection{The continuous expansion of activity objects in free based peer production communities}

\subsubsection{Integrated product extension}

\subsubsection{Product derivatives and variations}

\section{discussion}
\subsection{Domain specificities of software and hardware}

\subsection{Implications for design research and methodologies}

\subsection{Recommendations for further studies}

\section*{References}

\bibliographystyle{unsrtnat}

  %The Economic Phenomenon
  \bibitem{p2pEconomy}Bauwens, M., 2005. The political economy of peer production. CTheory, pp.12-1.
  \bibitem{vonHippel}Baldwin, C. and Von Hippel, E., 2011. Modeling a paradigm shift: From producer innovation to user and open collaborative innovation. Organization Science, 22(6), pp.1399-1417.
  \bibitem{p2pCapital}Bauwens, M., 2009. Class and capital in peer production. Capital & Class, 33(1), pp.121-141.

  %Design theory roles co-design


  %Peer production case studies

  \bibitem{RepRap}Kostakis, V. and Papachristou, M., 2014. Commons-based peer production and digital fabrication: The case of a RepRap-based, Lego-built 3D printing-milling machine. Telematics and Informatics, 31(3), pp.434-443.

  \bibitem{3dpp2p}Moilanen, J. and Vadén, T., 2013. 3D printing community and emerging practices of peer production. First Monday, 18(8).

  \bibitem{vonHippel_1}Franke, N., Von Hippel, E. and Schreier, M., 2006. Finding commercially attractive user innovations: A test of lead‐user theory. Journal of product innovation management, 23(4), pp.301-315.

  \bibitem{vonHippel_2}Von Hippel, Eric. "Lead users: a source of novel product concepts." Management science 32, no. 7 (1986): 791-805.

\end{thebibliography}
\section*{Acknowledgments}

Acknowledgments text here.

\appendix

\section*{Appendix}

Appendix text here.

\end{document}
