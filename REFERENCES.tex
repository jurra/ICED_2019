%The Economic Phenomenon
\bibitem{p2pEconomy}Bauwens, M., 2005. The political economy of peer production. CTheory, pp.12-1.

\bibitem{p2pCapital}Bauwens, M., 2009. Class and capital in peer production. Capital & Class, 33(1), pp.121-141.

%introduction
%cite
\bibitem{3d_printing}Rayna, T. and Striukova, L., 2016. From rapid prototyping to home fabrication: How 3D printing is changing business model innovation. Technological Forecasting and Social Change, 102, pp.214-224.

%cited
\bibitem{ExpansiveDesign}Engeström, Y., 2006. Activity theory and expansive design. Theories and practice of interaction design, pp.3-23.

%cited
\bibitem{CAD_review}Matta, A.K., Raju, D.R. and Suman, K.N.S., 2015. The integration of CAD/CAM and rapid prototyping in product development: a review. Materials Today: Proceedings, 2(4-5), pp.3438-3445.

%cite
\bibitem{CAD_Method}Meerkamm, H., 2011. Methodology and Computer-aided tools-a powerful interaction for product development. In The Future of Design Methodology (pp. 55-65). Springer, London.

%Design theory roles co-design

\bibitem{InventedHere}VICTOR, B., & BOYNTON, A. C. (1998). Invented here: maximizing your organization's internal growth and profitability. Boston, Mass, Harvard Business School Press.

%cited
\bibitem{hippel_2}Von Hippel, Eric. "Democratizing innovation: The evolving phenomenon of user innovation." Journal für Betriebswirtschaft 55, no. 1 (2005): 63-78.
%cited
\bibitem{hippel_1}Von Hippel, Eric. "Lead users: a source of novel product concepts." Management science 32, no. 7 (1986): 791-805.

\bibitem{hippel_3}Franke, N., Von Hippel, E. and Schreier, M., 2006. Finding commercially attractive user innovations: A test of lead‐user theory. Journal of product innovation management, 23(4), pp.301-315.

\bibitem{hippel_4}Baldwin, C. and Von Hippel, E., 2011. Modeling a paradigm shift: From producer innovation to user and open collaborative innovation. Organization Science, 22(6), pp.1399-1417.

\bibitem{bazaar}Raymond, E., 1999. The cathedral and the bazaar. Knowledge, Technology & Policy, 12(3), pp.23-49.

%Peer production case studies

\bibitem{RepRap}Kostakis, V. and Papachristou, M., 2014. Commons-based peer production and digital fabrication: The case of a RepRap-based, Lego-built 3D printing-milling machine. Telematics and Informatics, 31(3), pp.434-443.

\bibitem{3dpp2p}Moilanen, J. and Vadén, T., 2013. 3D printing community and emerging practices of peer production. First Monday, 18(8).

%Commons literature




% Distributed developments
\bibitem{Distributed_Development}Bird, C. and Nagappan, N., 2012, June. Who? where? what? examining distributed development in two large open source projects. In Mining Software Repositories (MSR), 2012 9th IEEE Working Conference on (pp. 237-246). IEEE.


%Open Source literature

\bibitem{FH}Free Hardware and Free Hardware Designs 2015, Richard Stallman, accessed 19 November 2018, <https://www.gnu.org/philosophy/free-hardware-designs.en.html>

\bibitem{what_OH} What is open hardware?,  opensource.com, accessed 19 November 2018,<https://opensource.com/resources/what-open-hardware>

\bibitem{OpenProductDeve}Riitahuhta, A., Lehtonen, T., Pulkkinen, A. and Huhtala, P., 2011. Open product development. In The Future of Design Methodology (pp. 135-146). Springer, London.

\bibitem{OSTangible}Raasch, C., Herstatt, C. and Balka, K., 2009. On the open design of tangible goods. R&d Management, 39(4), pp.382-393.


\bibitem{hackers}Lakhani, K.R. and Wolf, R.G., 2003. Why hackers do what they do: Understanding motivation and effort in free/open source software projects.

\bibitem{Outsourcing}Ågerfalk, P.J. and Fitzgerald, B., 2008. Outsourcing to an unknown workforce: Exploring opensurcing as a global sourcing strategy. MIS quarterly, pp.385-409.

\bibitem{OSProduct_Balka}

\bibitem{OSOrganizations}Crowston, K. and Scozzi, B., 2002. Open source software projects as virtual organisations: competency rallying for software development. IEE Proceedings-Software, 149(1), pp.3-17.

\bibitem{Adoption}Glynn, E., Fitzgerald, B. and Exton, C., 2005, November. Commercial adoption of open source software: an empirical study. In 2005 International Symposium on Empirical Software Engineering, 2005. (p. 10). IEEE.

\bibitem{HowItWorks}Lakhani, K.R. and Von Hippel, E., 2004. How open source software works:“free” user-to-user assistance. In Produktentwicklung mit virtuellen Communities (pp. 303-339). Gabler Verlag.

\bibitem{bazaar}  Demil, B. and Lecocq, X. (2006) ‘Neither Market nor Hierarchy nor Network: The Emergence of Bazaar Governance’, Organization Studies, 27(10), pp. 1447–1466. doi: 10.1177/0170840606067250.

\bibitem{Limux} Free software in government: Munich and LiMux 2017, Free Software Foundation, accessed 19 November 2017, <https://www.fsf.org/bulletin/2017/fall/free-software-in-government-munich-and-limux>

\bibitem{Freedoms} What is free software?, GNU, accessed 19 November 2017,
<https://www.gnu.org/philosophy/free-sw.en.html>

\bibitem{Million_Dollar}Torrone, P. and Fried, L., 2010. Million Dollar Baby. Businesses Designing and Selling Open Source Hardware, Making Millions. Talk at the O'Reilly foo camp east, 1.

%Beyond open source software

\bibitem{OH_Works?}Thompson, C., 2011. Build it. share it. profit. Can open source hardware work. Work, 10(08).

\bibitem{Para_shift}Maher, M., 1999. Open source software: The success of an alternative intellectual property incentive paradigm. Fordham Intell. Prop. Media & Ent. LJ, 10, p.619.

\bibitem{Economics}Lerner, J. and Tirole, J., 2005. The economics of technology sharing: Open source and beyond. Journal of Economic Perspectives, 19(2), pp.99-120.

\bibitem{EmergingProductiveModel}Kostakis, V., Niaros, V., Dafermos, G. and Bauwens, M., 2015. Design global, manufacture local: Exploring the contours of an emerging productive model. Futures, 73, pp.126-135.

\bibitem{digitalCommons}Kostakis, Vasilis, Kostas Latoufis, Minas Liarokapis, and Michel Bauwens. "The convergence of digital commons with local manufacturing from a degrowth perspective: two illustrative cases." Journal of Cleaner Production (2016).

\bibitem{OStangible}Raasch, C., Herstatt, C. and Balka, K., 2009. On the open design of tangible goods. R&d Management, 39(4), pp.382-393.

\bibitem{OSTangibleGoods}Balka, K., Raasch, C. and Herstatt, C., 2009, April. Open source beyond software: An empirical investigation of the open design phenomenon. In R&D Management Conference (pp. 14-16).

\bibitem{ActivityTheoryOpenSource}Hemetsberger, A. and Reinhardt, C., 2009. Collective development in open-source communities: An activity theoretical perspective on successful online collaboration. Organization studies, 30(9), pp.987-1008.

\bibitem{OSCar}

\bibitem{Rubow}Rubow, E., 2008. Open source hardware. Technical report.

\bibitem{Benkler}Benkler, Yochai. "Peer production and cooperation." Handbook on the Economics of the Internet 91 (2016).

\bibitem{PearceBusinessModel}Pearce, J.M., 2017. Emerging business models for open source hardware. Journal of Open Hardware.
