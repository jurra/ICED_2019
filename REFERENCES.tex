%The Economic Phenomenon
\bibitem{p2pEconomy}Bauwens, M., 2005. The political economy of peer production. CTheory, pp.12-1.

\bibitem{p2pCapital}Bauwens, M., 2009. Class and capital in peer production. Capital & Class, 33(1), pp.121-141.

%introduction
%cite
\bibitem{3d_printing}Rayna, T. and Striukova, L., 2016. From rapid prototyping to home fabrication: How 3D printing is changing business model innovation. Technological Forecasting and Social Change, 102, pp.214-224.

\bibitem{ExpansiveDesign}Engeström, Y., 2006. Activity theory and expansive design. Theories and practice of interaction design, pp.3-23.

%cited
\bibitem{CAD_review}Matta, A.K., Raju, D.R. and Suman, K.N.S., 2015. The integration of CAD/CAM and rapid prototyping in product development: a review. Materials Today: Proceedings, 2(4-5), pp.3438-3445.

%cite
\bibitem{CAD_Method}Meerkamm, H., 2011. Methodology and Computer-aided tools-a powerful interaction for product development. In The Future of Design Methodology (pp. 55-65). Springer, London.

%Design theory roles co-design

\bibitem{InventedHere}VICTOR, B., & BOYNTON, A. C. (1998). Invented here: maximizing your organization's internal growth and profitability. Boston, Mass, Harvard Business School Press. 

%cited
\bibitem{hippel_2}Von Hippel, Eric. "Democratizing innovation: The evolving phenomenon of user innovation." Journal für Betriebswirtschaft 55, no. 1 (2005): 63-78.
%cited
\bibitem{hippel_1}Von Hippel, Eric. "Lead users: a source of novel product concepts." Management science 32, no. 7 (1986): 791-805.

\bibitem{hippel_3}Franke, N., Von Hippel, E. and Schreier, M., 2006. Finding commercially attractive user innovations: A test of lead‐user theory. Journal of product innovation management, 23(4), pp.301-315.

\bibitem{hippel_4}Baldwin, C. and Von Hippel, E., 2011. Modeling a paradigm shift: From producer innovation to user and open collaborative innovation. Organization Science, 22(6), pp.1399-1417.

%Peer production case studies

\bibitem{RepRap}Kostakis, V. and Papachristou, M., 2014. Commons-based peer production and digital fabrication: The case of a RepRap-based, Lego-built 3D printing-milling machine. Telematics and Informatics, 31(3), pp.434-443.

\bibitem{3dpp2p}Moilanen, J. and Vadén, T., 2013. 3D printing community and emerging practices of peer production. First Monday, 18(8).

%Commons literature




% Distributed developments
\bibitem{Distributed_Development}Bird, C. and Nagappan, N., 2012, June. Who? where? what? examining distributed development in two large open source projects. In Mining Software Repositories (MSR), 2012 9th IEEE Working Conference on (pp. 237-246). IEEE.


%Open Source literature
\bibitem{OpenProductDeve}Riitahuhta, A., Lehtonen, T., Pulkkinen, A. and Huhtala, P., 2011. Open product development. In The Future of Design Methodology (pp. 135-146). Springer, London.


\bibitem{OSTangible}Raasch, C., Herstatt, C. and Balka, K., 2009. On the open design of tangible goods. R&d Management, 39(4), pp.382-393.

%Open Source Background

\bibitem{hackers}Lakhani, K.R. and Wolf, R.G., 2003. Why hackers do what they do: Understanding motivation and effort in free/open source software projects.
